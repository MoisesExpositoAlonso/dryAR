\documentclass[english]{article}
\usepackage[utf8]{inputenc}
\usepackage[T1]{fontenc}
\usepackage{babel}
\usepackage{amsmath}
\usepackage{graphicx}
\usepackage{fancyhdr}
\pagestyle{fancy}
\fancyhf{}
\renewcommand{\headrulewidth}{0pt}
\setlength{\headheight}{40pt} 
\lhead{\textsc{\scidatalogo}}
\rhead{\textsc{\overleaflogo}}

\begin{document}

\title{Data Descriptor Title (110 character maximum, inc. spaces)}

\author{Firstname Lastname\textsuperscript{1}, Firstname
Lastname\textsuperscript{2{*}}}

\maketitle
\thispagestyle{fancy}

1. An affiliation 2. A different affiliation {*}corresponding author(s):
Firstname Lastname (email@address)
\begin{abstract}
This is a manuscript template for Data Descriptor submissions to \emph{Scientific
Data} (http://www.nature.com/scientificdata). The abstract must be
no longer than 170 words, and should succinctly describe the study,
the assay(s) performed, the resulting data, and the reuse potential,
but should not make any claims regarding new scientific findings.
No references are allowed in this section. 
\end{abstract}

\section*{Background \& Summary}

(700 words maximum) An overview of the study design, the assay(s)
performed, and the created data, including any background information
needed to put this study in the context of previous work and the literature.
The section should also briefly outline the broader goals that motivated
the creation of this dataset and the potential reuse value. We also
encourage authors to include a figure that provides a schematic overview
of the study and assay(s) design. This section and the other main
body sections of the manuscript should include citations to the literature
as needed \cite{cite1, cite2}. References should be included within the 
manuscript file itself as our system cannot accept BibTeX bibliography files. 
Authors who wish to use BibTeX to prepare their references should therefore 
copy the reference list from the .bbl file that BibTeX generates and paste it 
into the main manuscript .tex file (and delete the associated 
\textbackslash{}bibliography and \textbackslash{}bibliographystyle commands).


\section*{Methods}

The Methods should include detailed text describing any steps or procedures 
used in producing the data, including full descriptions of the experimental 
design, data acquisition assays, and any computational processing (e.g. 
normalization, image feature extraction). Related methods should be grouped 
under corresponding subheadings where possible, and methods should be described 
in enough detail to allow other researchers to interpret and repeat, if required, 
the full study. Specific data outputs should be explicitly referenced via data 
citation (see Data Records and Data Citations, below). Authors should cite 
previous descriptions of the methods under use, but ideally the method 
descriptions should be complete enough for others to understand and reproduce 
the methods and processing steps without referring to associated publications. 
There is no limit to the length of the Methods section.

\subsection*{Code availability}

For all studies using custom code in the generation or processing of datasets, 
a statement must be included here, indicating whether and how the code can be 
accessed, including any restrictions to access. This section should also include 
information on the versions of any software used, if relevant, and any specific 
variables or parameters used to generate, test, or process the current dataset. 


\section*{Data Records}

Please explain each data record associated with this work, including
the repository where this information is stored, and an overview of
the data files and their formats. Each external data record should
be listed in Data Citation section at the end of this template, and 
records should be cited throughout the manuscript as, for example 
(Data Citation 1). 

Tables should be used to support the data records, and should clearly indicate 
the samples and subjects, their provenance, and the experimental manipulations 
performed on each. They should also specify the data output resulting from each 
data-collection or analytical step, should these form part of the archived record. 
Please see the submission guidelines at the \emph{Scientific Data} website, and 
our Word templates for more information on preparing such tables. 


\section*{Technical Validation}

This section presents any experiments or analyses that are needed
to support the technical quality of the dataset. This section may
be supported by up figures and tables, as needed. This is a required
section; authors must present information justifying the reliability
of their data.


\section*{Usage Notes}

Brief instructions that may help other researchers reuse these dataset.
This is an optional section, but strongly encouraged when helpful
to readers. This may include discussion of software packages that
are suitable for analyzing the assay data files, suggested downstream
processing steps (e.g. normalization, etc.), or tips for integrating
or comparing this with other datasets. If needed, authors are encouraged
to provide code, programs, or data processing workflows when they may help 
others analyse the data. We encourage authors to archive related code in 
a DOI-issuing archive when possible, but code may also be supplied as 
supplementary information files. 

For studies involving privacy or safety controls on public access
to the data, this section should describe in detail these controls,
including how authors can apply to access the data, and what criteria
will be used to determine who may access the data, and any limitations
on data use. 


\section*{Acknowledgements}

Text acknowledging non-author contributors. Acknowledgements should
be brief, and should not include thanks to anonymous referees and
editors, or effusive comments. Grant or contribution numbers may be
acknowledged. Author contributions Please describe briefly the contributions
of each author to this work on a separate line. 

AK did this and that. 

BG did this and that and the other. 


\section*{Competing financial interests}

A competing financial interests statement is required for all accepted
papers published in \emph{Scientific Data}. If none exist simply write,
``The author(s) declare no competing financial interests''.


\section*{Figures Legends}

Figure should be referred to using a consistent numbering scheme through
the entire Data Descriptor. For initial submissions, authors may choose
to supply this document as a single PDF with embedded figures, but
separate figure image files must be provided for revisions and accepted
manuscripts. In most cases, a Data Descriptor should not contain more
than three figures, but more may be allowed when needed. We discourage
the inclusion of figures in the Supplementary Information \textendash{}
all key figures should be included here in the main Figure section. 

Figure legends begin with a brief title sentence for the whole figure
and continue with a short description of what is shown in each panel,
as well as explaining any symbols used. Legend must total no more
than 350 words, and may contain literature references. 


\section*{Tables}

Tables supporting the Data Descriptor. These can provide summary information
(sample numbers, demographics, etc.), but they should generally not
be used to present primary data (i.e. measurements). Tables containing
primary data should be submitted to an appropriate data repository. 

Tables may be provided within the \LaTeX{} document or as separate
files (tab-delimited text or Excel files). Legends, where needed,
should be included here. Generally, a Data Descriptor should have
fewer than ten Tables, but more may be allowed when needed. Tables
may be of any size, but only Tables which fit onto a single printed
page will be included in the PDF version of the article (up to a maximum
of three). 

\begin{thebibliography}{1}
\expandafter\ifx\csname url\endcsname\relax
  \def\url#1{\texttt{#1}}\fi
\expandafter\ifx\csname urlprefix\endcsname\relax\def\urlprefix{URL }\fi
\providecommand{\bibinfo}[2]{#2}
\providecommand{\eprint}[2][]{\url{#2}}

\bibitem{cite1}
\bibinfo{author}{Califano, A.}, \bibinfo{author}{Butte, A.~J.},
  \bibinfo{author}{Friend, S.}, \bibinfo{author}{Ideker, T.} \&
  \bibinfo{author}{Schadt, E.}
\newblock \bibinfo{title}{{Leveraging models of cell regulation and GWAS data
  in integrative network-based association studies}}.
\newblock \emph{\bibinfo{journal}{Nature Genetics}}
  \textbf{\bibinfo{volume}{44}}, \bibinfo{pages}{841--847}
  (\bibinfo{year}{2012}).

\bibitem{cite2}
\bibinfo{author}{Wang, R.} \emph{et~al.}
\newblock \bibinfo{title}{{PRIDE Inspector: a tool to visualize and validate MS
  proteomics data.}}
\newblock \emph{\bibinfo{journal}{Nature Biotechnology}}
  \textbf{\bibinfo{volume}{30}}, \bibinfo{pages}{135--137}
  (\bibinfo{year}{2012}).

\end{thebibliography}

\section*{Data Citations}

Bibliographic information for the data records described in the manuscript.

1. Lastname1, Initial1., Lastname2, Initial2., ...\& LastnameN, InitialN. \emph{Repository name} Dataset accession number or DOI (YYYY).

\end{document}